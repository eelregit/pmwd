\documentclass[usenatbib]{mnras}

% Allow "Thomas van Noord" and "Simon de Laguarde" and alike to be sorted by "N" and "L" etc. in the bibliography.
% Write the name in the bibliography as "\VAN{Noord}{Van}{van} Noord, Thomas"
\DeclareRobustCommand{\VAN}[3]{#2}
\let\VANthebibliography\thebibliography
\def\thebibliography{\DeclareRobustCommand{\VAN}[3]{##3}\VANthebibliography}


\usepackage{microtype}
\usepackage{newpxtext, eulerpx}
\usepackage[T1]{fontenc}
\usepackage{amsmath}
\usepackage{bm}
\usepackage{graphicx}
\usepackage{booktabs}


\newcommand{\pmwd}{{\usefont{T1}{nova}{m}{sl}pmwd}}

\renewcommand{\sectionautorefname}{Sec.}
\renewcommand{\subsectionautorefname}{Sec.}
\renewcommand{\appendixautorefname}{App.}
\renewcommand{\figureautorefname}{Fig.}
%\newcommand{\subfigureautorefname}{\figureautorefname}

%FIXME copied from ../adjoint/adjoint.tex
\newcommand{\deltaD}{\delta^\textsc{d}}
\newcommand{\deltaK}{\delta^\textsc{k}}
\renewcommand{\d}{d}
\newcommand{\p}{\partial}
\newcommand{\cJ}{\mathcal{J}}
\newcommand{\cR}{\mathcal{R}}
\newcommand{\cL}{\mathcal{L}}
\newcommand{\cH}{\mathcal{H}}
\bmdefine{\vzero}{0}
\bmdefine{\vI}{I}
\bmdefine{\vnabla}{\nabla}
\bmdefine{\vtheta}{\theta}  % parameters
\bmdefine{\vomega}{\omega}  % white noise modes
\bmdefine{\vk}{k}  % wavevectors
\bmdefine{\vx}{x}  % comoving and canonical coordinates
\bmdefine{\vq}{q}  % Lagrangian coordinates
\bmdefine{\vs}{s}  % displacements
\bmdefine{\vp}{p}  % canonical momenta
\bmdefine{\va}{a}  % accelerations
\bmdefine{\vz}{z}  % states
\bmdefine{\vf}{f}
\bmdefine{\vF}{F}
\bmdefine{\vDelta}{\Delta}
\bmdefine{\vLambda}{\Lambda}
\bmdefine{\vlambda}{\lambda}
\bmdefine{\vvarphi}{\varphi}
\bmdefine{\vxi}{\xi}  % x adjoint
\bmdefine{\vpi}{\pi}  % p adjoint
\bmdefine{\valpha}{\alpha}  % force vjp x gradient
\bmdefine{\vzeta}{\zeta}  % force vjp theta gradient
\newcommand{\half}{\nicefrac12}
\newcommand{\As}{A_\mathrm{s}}
\newcommand{\ns}{n_\mathrm{s}}
\newcommand{\Omegam}{\Omega_\mathrm{m}}
\newcommand{\Omegab}{\Omega_\mathrm{b}}
\newcommand{\Mpc}{\mathrm{Mpc}}
\newcommand{\ic}{\mathrm{i}}
\newcommand{\Plin}{P_\mathrm{lin}}

\newcommand{\YL}[1]{\textcolor{Bittersweet}{#1}}



\title[Spatiotemporally Optimized Simulation]
{Spatiotemporal Optimization of Cosmological Particel-Mesh Simulation}


\author[Zhang, Li, Jamieson, et al.]{
%
Yucheng Zhang,$^{1, 2}$
%
Yin Li,$^{1, 3}$\thanks{Email: XYZ, eelregit@gmail.com, and XYZ}
%
and Drew Jamieson$^{4}$
%
\\$^1$Department of Mathematics and Theory, Peng Cheng Laboratory,
Shenzhen, Guangdong 518066, China
%
\\$^2$Center for Cosmology and Particle Physics, Department of Physics,
New York University, New York, NY 10003, USA
%
\\$^3$Center for Computational Astrophysics \& Center for Computational
Mathematics, Flatiron Institute, New York, NY 10010, USA
%
\\$^4$Max Planck Institute for Astrophysics, 85748 Garching bei
M\"unchen, Germany
}


\date{Accepted XXX. Received YYY; in original form ZZZ}
\pubyear{2023}



\begin{document}
\label{firstpage}
\pagerange{\pageref{firstpage}--\pageref{lastpage}}
\maketitle



\begin{abstract}
We sharpen PM force and optimize integration time steps.
\end{abstract}

\begin{keywords}
cosmology: large-scale structure of Universe
-- methods: numerical
-- software: development
\end{keywords}



\section{Introduction}


\section{Methods}


Given two snapshots at $a_i$ and $a_{i+1}$, we get the particle displacements
in between at $a$ using interpolation with the cubic Hermite spline,
\begin{align}
  \bm{x}(a) =\ &h_{00}(\alpha)\bm{x}_i + h_{10}(\alpha)(a_{i+1} - a_i)\bm{v}_i + \nonumber\\
           &h_{01}(\alpha)\bm{x}_{i+1} + h_{11}(\alpha)(a_{i+1} - a_i)\bm{v}_{i+1} \,,
\end{align}
where $\alpha := (a - a_i)/(a_{i+1} - a_i)$ and the Hermite basis functions read
\begin{align}
  h_{00}(\alpha) &= 2\alpha^3 - 3\alpha^2 + 1 \,,\nonumber\\
  h_{10}(\alpha) &= \alpha^3 - 2\alpha^2 + \alpha \,,\nonumber\\
  h_{01}(\alpha) &= -2\alpha^3 + 3\alpha^2 \,,\nonumber\\
  h_{11}(\alpha) &= \alpha^3 - \alpha^2 \,.
\end{align}
The particle velocity is then given by the derivative $d\bm{x}/da$.


\subsection{Spatial Optimization}


\subsection{Temporal Optimization}



\begin{table*}
  \centering
  \caption{Ranges of GADGET-4 and \pmwd\ configuration and cosmological
    parameters.
    Note that the grid ratio need next fast len to determine the mesh
    shape for fast FFT.
    Given the box size, the mesh shape determines the cell size.
    The softening parameter gives the ratio of the comving softening
    length to the mean particle spacing.
    The curvature $\Omega_k$ is related to the separate universe
    simulation.
    We sample parameters applicable to GADGET-4 during data generation,
    and those applicable to \pmwd\ during training.
    \textsuperscript\dag The box size is determined jointly by the
    \pmwd\ mesh shape and mesh cell size below, which are assumed to be
    sampled independently.
    In practice, we sample one conditioned on the other and the box
    size.
    * Number of time steps from beginning to end ...
  }
  \label{tab:param}
  \begin{tabular}{lcccr}
    \toprule
    applicability & parameter & distribution & range & sampling \\
    \midrule
    & $128^3$ white noise & $\mathcal{N}(\vzero, \vI)$ & 512 realizations & MC \\
    \cmidrule(lr){2-5}
    & 121 snapshots, $a$ & uniform & [1/16, 1] & deterministic \\
    \cmidrule(lr){2-5}
    & box size in Mpc\textsuperscript\dag & log-trapezoidal & [25.6, 2560) \\
    \cmidrule(lr){2-4}
    & snapshot offset, $\Delta\!a$ & uniform & [0, 1/128) \\
    \cmidrule(lr){2-4}
    GADGET-4 & $A_\mathrm{s} \times 10^9$ & log-uniform & [1, 4) \\
    \cmidrule(lr){2-4}
    \& \pmwd\ & $n_\mathrm{s}$ & log-uniform & [0.75, 1.25) & RQMC \\
    \cmidrule(lr){2-4}
    & $\Omega_\mathrm{m}$ & log-uniform & [1/5, 1/2) & see \autoref{fig:sobol} \\
    \cmidrule(lr){2-4}
    & $\Omega_\mathrm{b} / \Omega_\mathrm{m}$ & log-uniform & [1/16, 1/2) \\
    \cmidrule(lr){2-4}
    & $\Omega_k / (1 - \Omega_k)$ & uniform & [-0.5, 0.5) \\
    \cmidrule(lr){2-4}
    & $h$ & log-uniform & [0.5, 1) \\
    \cmidrule(lr){1-4}
    GADGET-4 & softening ratio & log-uniform & [1/50, 1/20) \\
    \cmidrule(lr){1-5}
    & mesh shape & log-uniform & [128, 512] \\
    \cmidrule(lr){2-4}
    \pmwd\ & mesh cell size in Mpc & log-uniform & [0.2, 5] & MC \\
    \cmidrule(lr){2-4}
    & number of time steps* & log-uniform & [10, 1000] \\
    \bottomrule
  \end{tabular}
\end{table*}


\subsection{Parameters and Configurations}

We run 512 GADGET-4 simulations that cover a wide range of
configurations and cosmological parameters, as listed in
\autoref{tab:param}.
Each simulation has $128^3$ particles and writes 121 snapshots linearly
spaced in scale factor from $1/16 + \Delta a$ to $1 + \Delta a$, with
spacing $1/128$.
$\Delta a$ is an offset, common for all snapshots in each simulation,
that varies as in \autoref{tab:param} and helps to sample different
alignment between snapshot and start/stop times.

The curvature $\Omega_k$ is related to the separate universe simulation
\citep{LiEtAl2014a, WagnerEtAl2015}.


\begin{figure*}
  \centering
  \includegraphics[width=0.9\linewidth]{sobol.pdf}
  \caption{Randomized Quasi-Monte Carlo (RQMC) configuration with
    scrambled Sobol sequence of 512 points in 9D.
    Lower triangular panels show the 2D projections and the diagonal
    panels are the 1D cumulative histograms.
    From left to right (top to bottom), we use each dimension of the
    sample to scale the parameters as ordered in \autoref{tab:param}.
    We use the \texttt{scipy.stats.qmc} package \citep{SciPy} to
    generate the Sobol sequence \citep{Sobol1967}, which uses the
    direction number from \citet{JoeKuo2008} and the Owen scrambling
    \citep{Owen1998}.
    We search among 65536 scrambling seeds to minimize the mixture
    discrepancy (a uniformity measure) proposed in \citet{Zhou2013MD}.
  }
  \label{fig:sobol}
\end{figure*}

\begin{figure*}
  \centering
  %\includegraphics[width=0.8\linewidth]{name.ext}
  \caption{1D histogram of the parameters sampled with RQMC and scaled
  as in \autoref{tab:param}}.
  \label{fig:hist}
\end{figure*}


\subsection{Simulations}

\citet{GADGET-4}

We configure GADGET-4 with high force and time integration accuracy
settings.
For gravitational force, we choose pure FMM with $p=5$ and opening angle
0.4.
%FIXME list other from Config.sh and param.txt


\section{Results}


\section{Conclusions}


\section*{Acknowledgements}

The Flatiron Institute is supported by the Simons Foundation.


\section*{Code Availability}

\pmwd\ is open-source on
\href{https://github.com/eelregit/pmwd}{GitHub}, including the
\href{https://github.com/eelregit/pmwd/tree/master/docs/papers/sto}{source
and scripts of this paper}.



\bibliographystyle{mnras}
\bibliography{sto}



\bsp  % typesetting comment
\label{lastpage}
\end{document}
