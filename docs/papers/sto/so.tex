\documentclass[modern, trackchanges, dvipsnames]{aastex631}
  \urlstyle{sf}


\usepackage{CJK}
\usepackage{microtype}
\usepackage{savesym}
  \savesymbol{lambdabar}
\usepackage{newpxtext, eulerpx}
  \restoresymbol{newpx}{lambdabar}
\usepackage[T1]{fontenc}
\usepackage{fontawesome}
\usepackage{amsmath}
\usepackage{bm}
\usepackage{nicefrac}
\usepackage[caption=false]{subfig}
\usepackage[figure,figure*]{hypcap}
%\usepackage{tikz}
%  \usetikzlibrary{positioning, fit, calc, arrows.meta}
\usepackage{booktabs}


\newcommand{\pmwd}{{\usefont{T1}{nova}{m}{sl}pmwd}}
\newcommand{\GADGET}{{{\fontsize{10pt}{12pt}\selectfont GADGET}-4}}

%\renewcommand{\sectionautorefname}{Sec.}
%\renewcommand{\subsectionautorefname}{Sec.}
%\renewcommand{\appendixautorefname}{App.}
%\renewcommand{\figureautorefname}{Fig.}
%\newcommand{\subfigureautorefname}{\figureautorefname}

%FIXME copied from ../adjoint/adjoint.tex
\newcommand{\deltaD}{\delta^\textsc{d}}
\newcommand{\deltaK}{\delta^\textsc{k}}
\renewcommand{\d}{d}
\newcommand{\p}{\partial}
\newcommand{\cJ}{\mathcal{J}}
\newcommand{\cR}{\mathcal{R}}
\newcommand{\cL}{\mathcal{L}}
\newcommand{\cH}{\mathcal{H}}
\bmdefine{\vzero}{0}
\bmdefine{\vI}{I}
\bmdefine{\vnabla}{\nabla}
\bmdefine{\vtheta}{\theta}  % parameters
\bmdefine{\vvartheta}{\vartheta}  % relevant scales and factors to form
                                  % dimensionless features as NN inputs
\bmdefine{\vomega}{\omega}  % white noise modes
\bmdefine{\vk}{k}  % wavevectors
\bmdefine{\vx}{x}  % comoving and canonical coordinates
\bmdefine{\vq}{q}  % Lagrangian coordinates
\bmdefine{\vs}{s}  % displacements
\bmdefine{\vp}{p}  % canonical momenta
\bmdefine{\vv}{v}  % FIXME some velocities
\bmdefine{\va}{a}  % accelerations
\bmdefine{\vz}{z}  % states
\bmdefine{\vo}{o}  % observable
\bmdefine{\vf}{f}
\bmdefine{\vF}{F}
\bmdefine{\vDelta}{\Delta}
\bmdefine{\vLambda}{\Lambda}
\bmdefine{\vlambda}{\lambda}
\bmdefine{\vvarphi}{\varphi}
\bmdefine{\vxi}{\xi}  % x adjoint
\bmdefine{\vpi}{\pi}  % p adjoint
\bmdefine{\valpha}{\alpha}  % force vjp x gradient
\bmdefine{\vzeta}{\zeta}  % force vjp theta gradient
\newcommand{\lna}{\ln\!a}
\newcommand{\lnk}{\ln\!k}
\newcommand{\half}{\nicefrac12}
\newcommand{\As}{A_\mathrm{s}}
\newcommand{\ns}{n_\mathrm{s}}
\newcommand{\Omegam}{\Omega_\mathrm{m}}
\newcommand{\Omegab}{\Omega_\mathrm{b}}
\newcommand{\Mpc}{\mathrm{Mpc}}
\newcommand{\ic}{\mathrm{i}}
\newcommand{\linear}{\mathrm{lin}}
\newcommand{\tophat}{\mathrm{TH}}
\newcommand{\gauss}{\mathrm{G}}

\newcommand{\GPU}{NVIDIA H100 PCIe}  % 80GB memory

\newcommand{\YL}[1]{\textcolor{Bittersweet}{#1}}
\newcommand{\YZ}[1]{\textcolor{MidnightBlue}{#1}}



\begin{document}



\title{\large Neural, Tree, and Symbolic Optimization of Cosmological
Particle-Mesh Simulation
\vspace{0.3em}}


\author[0000-0002-9300-2632]{\normalsize Yucheng Zhang}
\affiliation{Department of Mathematics and Theory, Peng Cheng
Laboratory, Shenzhen, Guangdong 518066, China}
\affiliation{Center for Cosmology and Particle Physics, New York
University, New York, New York 10003, USA}

\author[0000-0002-0701-1410]{\normalsize Yin Li}
\affiliation{Center for Computational Mathematics, Flatiron Institute,
New York, New York 10010, USA}
\affiliation{Department of Mathematics and Theory, Peng Cheng
Laboratory, Shenzhen, Guangdong 518066, China}
\affiliation{Center for Computational Astrophysics, Flatiron Institute,
New York, New York 10010, USA}

\author[0000-0001-5044-7204]{\normalsize Drew Jamieson}
\affiliation{Max Planck Institute for Astrophysics, 85748 Garching bei
M\"unchen, Germany}

\author[0000-0003-0745-9431]{\normalsize Libin Lu}
\affiliation{Center for Computational Mathematics, Flatiron Institute,
New York, New York 10010, USA}


\shorttitle{Optimized Cosmological Simulation}
\shortauthors{Zhang \& Li et al.}


\correspondingauthor{XYZ, Yin Li, and XYZ}
\email{XYZ, eelregit@gmail.com, and XYZ}



\begin{abstract}

We sharpen the PM force with first neural networks and then symbolic
regression, taking symmetry and dimensional analysis in consideration.

We combine differentiable simulation, deep learning, [decision tree
(feature selection)], and symbolic regression to extract the final kernel
applicable to all other cosmological simulation code.

\end{abstract}



\section*{TODO}
\begin{itemize}
\item (paper) Moving some of the following ongoing and completed tasks
  to the main text and add preliminary figures.
\item (data; YL) Incorporate 3lpt branch; testing; find a\_start for
  different mass resolutions; probably replace linear snapshot spacing
  according to the nonlinearity (estimate (with PM till convergence) the
  std acc over the std vel) of each realization.
\item (model; YL) Change nbody for loop by lax scan and/or while loops.
  while\_loop can help to save compilation when trained on different
  snapshots. Performance difference between scan and while?
\item (model; YL) Compare one pair of \GADGET\ and \pmwd\ (with say 100 to 1000 time
    steps) results and make sure that they agree perfectly on the large
    scales. For this purpose the box size should be large enough (in
    linear regime), say 1 Gpc/$h$.\newline
    \YZ{\textbf{Update}: for Sobol=6, boxsize=983Mpc, \GADGET\ and \pmwd\ are
    visually consistent at the field level. For power spectrum: For
    odd mesh shapes (e.g. 1, 3, 5), pmwd and \GADGET\ are perfectly consistent
    on large scales. \textbf{While for even mesh shapes (e.g. 2, 4), the powers
    look slightly different. To figure out this weird issue.}}
\item (model; YL) Probably necessary to first incorporate List \& Hahn's
  recent improvement on FastPM integrator arXiv:2301:09655.
\item (train, test) For $\vs(\vq)$, $\vv(\vq)$, $\delta(\vx)$,
  $(1+\delta)\vv(\vx)$, $(1+\delta)\vv\vv(\vx)$, show $T - 1$ in symlog
  with 0.01 linthresh, and $1-r^2$ in log at 3 different $a$'s; Show $f$
  and $g$ (along axis and face diagonal) at 3 different $a$'s; Track
  mean epoch loss
\item (train) Randomize snapshots of each simulation instead of
  sequential; alternatively accumulate their gradients; Normalize loss
  by number of time steps\newline
  \YZ{\textbf{Update}: Normalizing loss by number of time steps seems to halt
  the training of late time snapshots, i.e. they stop improving.}
\item (train, test) 3 sets of grid for ptcl, force mesh, and loss. Add
  random offset in loss grid (see \autoref{tab:param}) to remove the
  position dependence in loss; likewise for force mesh? how about only
  training (for lower loss), only ``inference'' (for higher accuracy),
  or both? the random force mesh offset can also be replaced by a
  structured scheme, e.g., alternating between 2 interlocking grids
\item (train) Explore if we can use a finer loss grid, from 1x to say
  1.5x and even 2x.
\item (train, test) Non-integer multiple mesh shapes: for uniform
  particle grid, test (with double precision on CPUs) whether they start
  moving by artifact forces on cell scales, in 3 cases: no
  deconvolution, with CIC deconvolution, with SO.
\item (train) find the best loss function for production training
\item (train) For generalizability and robustness, consider testing
  weight decay and dropout. Also they're worth adding if they give lower
  losses at convergence. Note that 512 realization are not too many, and
  we have features that can be very correlated.\newline
  \YZ{Dropout with Flax has been coded. We can use Optax AdamW for weight decay.}
\item (train) optuna hyperparam search on a subset of training data
  before production training
  (\href{https://github.com/optuna/optuna-examples/blob/main/haiku/haiku_simple.py}{optuna
  example}).\newline
  \YZ{Do we also use another subset of the training data as validation data?}
\item (train) Reduce learning rate on (mean epoch) loss plateau in
  production\newline
  \YZ{A simple scheduler has been coded, to be fine-tuned for production training.}
\item (test) Interpolation test: compare SO DM and halo fields and
  summary stats with those of raw pmwd, deconvolved pmwd, and \GADGET,
  on the last 9 test cases and noncubic box.
\item (test) Extrapolation test: demonstrate generalization capability,
  compare extrap PySR, extrap SO, raw pmwd, deconv pmwd, and \GADGET.
\item (test) Generate a sample of $(X, y)$ data with the trained neural network
  for feature selection and symbolic regression.
\item (paper) Think about what figures and tables to show in the paper. Update
to 6 items could be included. Some tables could be included in the context as
list if we have too many figures to show?\newline
  \YZ{figures:
  1. $f$ and $g$ functions, both NN and Symbolic.
  2. Density field (and halo?) with zoom in for small scale sharpening.
  3. Transfer function and correlation coefficient.
  4. Halo related stats.}
\item Fix cross references and other journal quirks as a patch
\end{itemize}


\vspace{1em}
\section{Introduction}


\vspace{1em}
\section{Results}
% Results should be divided by topical subheadings

\subsection{Trained SO neural networks and symbolic expressions}


\subsection{Dark matter fields}


\subsection{Halo catalogs}


\vspace{1em}
\section{Discussion}
% Discussion does not contain subheadings


\vspace{1em}
\section{Methods}


\vspace{1em}
\subsection{Differentiable Simulation}

The differentiable cosmological simulation with adjoint method \citep{Li2022a}
and the corresponding PM $N$-body library \pmwd\ \citep{Li2022b} are the
foundation of the optimization discussed in this work.
We denote the particles' state with $\vz = (\vx, \vp)^\intercal$ and their
adjoint variables with $\vlambda = (\vxi, \vpi)$.



\vspace{1em}
\subsection{Snapshot Observable}
\label{sec:snapobs}

Given two snapshots at $a_i$ and $a_{i+1}$, we get the particle displacements
in between at $a$ using interpolation with the cubic Hermite spline,
\begin{align}
  \vx(a) =\ &h_{00}(\alpha)\vx_i + h_{10}(\alpha)(a_{i+1} - a_i)\vv_i + \nonumber\\
           &h_{01}(\alpha)\vx_{i+1} + h_{11}(\alpha)(a_{i+1} - a_i)\vv_{i+1},
\end{align}
where $\alpha := (a - a_i)/(a_{i+1} - a_i)$ and the Hermite basis functions read
\begin{align}
  h_{00}(\alpha) &= 2\alpha^3 - 3\alpha^2 + 1, \nonumber\\
  h_{10}(\alpha) &= \alpha^3 - 2\alpha^2 + \alpha, \nonumber\\
  h_{01}(\alpha) &= -2\alpha^3 + 3\alpha^2, \nonumber\\
  h_{11}(\alpha) &= \alpha^3 - \alpha^2.
\end{align}
The particle velocity is then given by the derivative $\d\vx / \d a$.

We refer to the quantities that are directly related to the objective function
as the observables.
Our observable $\vo$ is a snapshot $\hat\vz = (\hat\vx,
\hat\vp)^\intercal$ at a given time $a_o$, which in general can be given by the
interpolation of two snapshots $\vz_{k-1}$ and $\vz_k$ where $a_{k-1} \leq a_o <
a_k$.
Then ${\p\vo}/{\p\vz_i}$ would be zero except for the relevant $\vz_{k-1}$
and $\vz_k$. We have in the initial conditions of the adjoint equations
\begin{align}
  \vxi_n &= \frac{\p\cJ}{\p\vo} \cdot \frac{\p\vo}{\p\vx_n}, \nonumber\\
  \vpi_n &= \frac{\p\cJ}{\p\vo} \cdot \frac{\p\vo}{\p\vp_n}.
\end{align}
In the reverse time evolution of the adjoint equations, these adjoint varibales
remain zero until the time stepping from $a_{k+1}$ to $a_k$.


\vspace{1em}
\subsection{Optimization of Dynamics}
\label{sec:so}

In Fourier space, the gravitational force field reads
\begin{equation}
- i \vk \varphi(\vk) = i \frac32 \Omegam H_0^2 \,\frac{\vk}{k^2} \delta(\vk),
\end{equation}
Force sharpening
%
\begin{equation}
\frac{i k_i}{k^2} \to \frac{i k_i}{k^2}
  f(k_i; \vvartheta) g(k_1, k_2, k_3; \vvartheta),
\end{equation}
%
where $f$, $g$ are neural networks, each approximating a nonlinear
function of the wavevector components and all possible relevant features
$\vvartheta$ summarized in \autoref{tab:feat} and discussed below.

\begin{table}
\centering
\caption{TODO Possible features that can affect force sharpening.
All scales are then multiplied by the wavenumber $k$ to form dimensionless
factors.}
\label{tab:feat}
\begin{tabular}{rl}
\toprule
type & feature \\
\midrule
& $R_\mathrm{d}$: RMS linear displacement \\
\cmidrule(lr){2-2}
& $R_P$: $\Delta_\linear^2(R_P^{-1}) \triangleq 1$ \\
\cmidrule(lr){2-2}
nonlinear scales & $R_\tophat$: $\sigma_\tophat^2(R_\tophat) \triangleq 1$ \\
\cmidrule(lr){2-2}
& $R_\gauss$: $\sigma_\gauss^2(R_\tophat) \triangleq 1$ \\
\cmidrule(lr){2-2}
& $\d R / \d\lna$: time derivatives of the above \\
\midrule
& particle spacing \\
\cmidrule(lr){2-2}
other scales & $l$: PM force mesh cell size \\
\cmidrule(lr){2-2}
& softening length \\
\midrule
& $G_m(a) \triangleq D_m(a) / a^m, m \in \{1, 2, \YL{3's?}\}$ \\
\cmidrule(lr){2-2}
& $\d\ln G_m / \d\lna$ \\
\cmidrule(lr){2-2}
dimensionless factors & $\Omegam(a)$ \\
\cmidrule(lr){2-2}
& $\d\ln\!H / \d\lna$ \\
\cmidrule(lr){2-2}
& $\Delta\lna$: time step size in $\lna$ \\
\bottomrule
\end{tabular}
\end{table}

The dimensionless power spectrum
%
\begin{equation}
\Delta^2(k) \triangleq \frac{k^3 P(k)}{2 \pi^2}.
\end{equation}

Variance of linear matter overdensity in a top-hat window
%
\begin{equation}
\sigma_\tophat^2(R) = \int_0^\infty \frac{\d k}k
  \Delta_\linear^2(k) W_\tophat^2(kR),
\end{equation}
where $W_\tophat(kR) = 3[\sin(kR) - kR\cos(kR)] / (kR)^3$ is the
Fourier transform of the top-hat window.
Likewise $\sigma_\gauss^2(R)$ is defined with a Gaussian window
$W_\gauss(kR) = e^{-(kR)^2/2}$.

The RMS linear displacement
%
\begin{equation}
R_\mathrm{d} = \int \frac{k P_\linear}{6\pi^2} \d\lnk.
\end{equation}
%

Relevant scale-independent dimensionless factors are
\begin{itemize}
\item $G_m(a) \triangleq D_m(a) / a^m$: linear growth given in
  suppression factors for $m \in \{1, 2\}$;
\item $\d\ln G_m / \d\lna$: with $m=1$ case related to the growth rate
  $f \triangleq \d\ln D_1 / \d\lna$;
\item $\Omegam(a)$: matter density parameter at $a$;
\item $\d\ln\!H / \d\lna$: time derivative of Hubble expansion;
\item $\Delta\lna$: time step size in $\lna$.
\end{itemize}

We combine the above list of parameters in $\vvartheta$.

Dimensionless combinations, e.g.,
%
\begin{equation}
f(k_i; \vvartheta) = f(k_iR_d, k_iR_\tophat, \cdots; G_m, \Omegam, \cdots)
\end{equation}
%
\begin{equation}
g(k_1, k_2, k_3; \vvartheta) = g(\{k_jR_d, k_jR_\tophat, \cdots\}_{j=1,2,3}; G_m, \Omegam, \cdots)
\end{equation}
%
$f$ and $g$ are optimized during training to correct and sharpen the
gravitational forces.
Gravity obeys the $\mathrm{O}(3)$ rotational symmetry, which is however
broken by the PM solver to the O\textsubscript{h} discrete symmetry with
48 elements involving permutations and reversals of the 3 principal axes
of the simulation box.
The PM evaluation of each force component can further break the symmetry
along that axis, leaving the remaining 2 permutationally symmetric.
Therefore, we implement $f$ and $g$ to be even functions of the
wavevector components, and $g$ to be permutation equivariant in all 3
components, so that $f g$ is the most general modification to the force
kernel.
An even and real kernel in configuration space is also even in Fourier
space.
We take the absolute value for the evenness, and sort the 3 wavevector
compontents to achieve the permutation equivariance.

\YZ{
The parameters of the neural networks, which enter every force evaluation in
every time integration step, can be treated just the same as other cosmological
parameters.
The objective gradients are accumulated, based on which the parameters are then
optimized.
For both $f$ and $g$ neural networks, we adopt the multilayer perceptron (MLP)
structure, each with one input layer, two hidden layers and one output layer.
}


\vspace{1em}
\subsection{Loss Function}

\YL{
For a field $f$ in either Lagrangian or Eulerian space, we have thought
of 3 forms of MSE-inspired loss:
%
\begin{equation}
\cL_f = \frac1{N_a} \ln
  \frac{\sum_\vq \bigl[ \hat f(\vq) - f(\vq) \bigr]^2}
       {\sum_{\vq'} f(\vq')^2},
\end{equation}
%
where $N_a$ is the time steps it take to generate the snapshot for each
iteration.
(or with $\vq$ replaced by $\vx$ for Eulerian fields, or equivalently by
their Fourier wavevector $\vk$ in both cases)
%
\begin{equation}
\cL_f = \frac1{N_a N_K} \sum_i \ln
\frac{\sum_{k \in K_i} \bigl| \hat f(\vk) - f(\vk) \bigr|^2}
     {\sum_{k' \in K_i} | f(\vk')|^2},
\end{equation}
%
and
%
\begin{equation}
\cL_f = \frac1{N_a} \ln \biggl[ \frac1{N_K} \sum_i
\frac{\sum_{k \in K_i} w(\vk)
      \bigl| \hat f(\vk) - f(\vk) \bigr|^2}
     {\sum_{k' \in K_i} |f(\vk')|^2} \biggl].
\end{equation}
%
Then the question is which loss we should use for which $f$, to get
stable and robust results, e.g., no bump for large or small boxes.
To further limit our scope in this paper, we should probably only
consider $f \in \{\vs, \delta\}$, and optionally the Lagrangian particle
velocity $\vv(\vq)$.
}

For the displacement field, we adopt the mean squared error (MSE) loss
that proved effective in many previous works
\citep[e.g.,][]{HeEtAl2019, LiEtAl2021}:
%
\begin{equation}
\cL_\vs = \ln \frac{\sum_\vq \bigl[ \hat\vs(\vq) - \vs(\vq) \bigr]^2}
                   {\sum_{\vq'} s(\vq')^2},
\end{equation}
%
in which a particle originating from its Lagrangian position $\vq$ is
displaced by $\vs$ in the \GADGET\ snapshot and by $\hat\vs$ in the
prediction of \pmwd\ model in training.
Note that we have further taken the logarithm to combine it with the
other loss(es) below.
% Say that only s loss is not enough even though in principle it is, and cite DrewEtAl x2 too

\YL{Even though the log should be enough for adaptive optimizer and
fixed sim config, this may create problem when training on all config at
the same time. So normalizations inside the log is added above and
below.}

However, in the Eulerian space, an MSE loss is dominated by the small
scale modes.
In this work we choose uniform weights in the scale space on the
relative MSE.
We bin the modes logarithmically by \nicefrac13 octave in scale, with
the $i$-th $k$-bin $K_i \triangleq [k_i, k_{i+1})$, where $k_i =
2^{1+i/3} \pi / L$.
Our loss on the overdensity field $\delta$ sums up its MSE relative to
the target field power in logarithmic bins:
%
\begin{equation}
\cL_\delta = \frac1{N_K} \sum_i \ln
\frac{\sum_{k \in K_i} \bigl| \hat\delta(\vk) - \delta(\vk) \bigr|^2}
     {\sum_{k' \in K_i} |\delta(\vk')|^2},
\end{equation}
%
\begin{equation}
\cL_\delta = \frac1{N_K} \sum_i
\frac{\sum_{k \in K_i} w(\vk)
      \bigl| \hat\delta(\vk) - \delta(\vk) \bigr|^2}
     {\sum_{k' \in K_i} |\delta(\vk')|^2},
\end{equation}
%
where $N_K$ is the number of $k$-bins and $w(\vk)$ is a nonnegative
factor that suppresses the high $k$ contribution in Eulerian fields such
as the density field (with $w=1$ for the Lagrangian displacement field).
This is because a PM solver gradually loses small scale information step
by step, and without $w$ setting the priority, the loss function could
encourage the network to fit the small scales at the cost of the
intermediate scales.

To understand the form of the loss function, let's introduce for each
mode the square root ratio of power
%
\begin{equation}
T(\vk) \triangleq
\sqrt{\frac{|\hat f(\vk)|^2}{|f(\vk)|^2}},
\end{equation}
%
and the cross correlation coefficient
%
\begin{equation}
r(\vk) \triangleq
\frac{\Re \bigl[ \hat f(\vk) f^*(\vk) \bigr]}
     {\sqrt{|\hat f(\vk)|^2 |f(\vk)|^2}}.
\end{equation}
%
Now, under the mild assumptions that $T$ and $r$ are smooth and
isotropic, we can show
%
\begin{equation}
\cL_\delta \propto \int \!\d\ln\!k\, \ln
\Bigl\{ \bigl[ 1 - T(\vk) \bigr]^2
  + 2 T(\vk) \bigl[ 1 - r(\vk) \bigr] \Bigr\} \geq -\infty,
\end{equation}
%
replacing
%
\begin{equation}
\cL_\delta \propto \int \!\d\ln\!k\, w(\vk)
\Bigl\{ \bigl[ 1 - T(\vk) \bigr]^2
  + 2 T(\vk) \bigl[ 1 - r(\vk) \bigr] \Bigr\} \geq 0,
\end{equation}
%
that only vanishes when both $T$ and $r$ are unity for all modes, i.e.,
when $\hat\delta$ is identical to $\delta$.
Above we choose the binned version instead for the robustness against
the noisy modewise $|\delta(\vk)|^2$ in the denominator.
The derivation here is similar to that in \citet{HeEtAl2019} on the MSE
loss, while our modification is important to balance among all scales so
that the neural network does not focus too much on either high or low
$k$'s.

We combine Lagrangian and Eulerian fields to form the total loss:
%
\begin{equation}
\cJ = \cL_\vs + [\cL_\vv] + \cL_\delta.
\end{equation}
%

\YL{TODO move this above:}
Note that $\cL_\vs$ is on the displacement field $\vs(\vq)$ in the
Lagrangian space, while the density field is in the Eulerian space:
%
\begin{equation}
1 + \delta(\vx) = \int\!\d^3\vq\, \deltaD[ \vx - \vq - \vs(\vq)],
\end{equation}
%
which we compute by scattering particles to the choice of grid for loss.


\begin{table*}
\centering
\caption{Ranges of \GADGET\ and \pmwd\ configuration and cosmological
parameters.
Note that the grid ratio need next fast len to determine the mesh shape
for fast FFT.
Given the box size, the mesh shape determines the cell size.
The softening parameter gives the ratio of the comoving softening length
to the mean particle spacing.
The curvature $\Omega_k$ is related to the separate universe simulation.
We sample parameters applicable to \GADGET\ during data generation, and
those applicable to \pmwd\ during training.
\textsuperscript\dag{}The box size is determined jointly by the \pmwd\
mesh shape and mesh cell size below, which are assumed to be sampled
independently. \YL{I forgot why I wrote ``Mpc'' without the $h$ here.}
In practice, we sample one conditioned on the other and the box size.
*Number of time steps from $a=1/16$ to $a=1$.
}
\label{tab:param}
\begin{tabular}{lcccr}
\toprule
applicability & parameter & distribution & range & sampling \\
\midrule
& $128^3$ white noise & $\mathcal{N}(\vzero, \vI)$ & 512 realizations & MC \\
\cmidrule(lr){2-5}
& 121 snapshots, $a$ & uniform & [1/16, 1] & deterministic \\
\cmidrule(lr){2-5}
& box size in Mpc\textsuperscript\dag & log-trapezoidal & [25.6, 2560) \\
\cmidrule(lr){2-4}
& snapshot offset, $\Delta\!a$ & uniform & [0, 1/128) \\
\cmidrule(lr){2-4}
\GADGET\ & $\As \times 10^9$ & log-uniform & [1, 4) \\
\cmidrule(lr){2-4}
\& \pmwd\ & $\ns$ & log-uniform & [0.75, 1.25) & RQMC \\
\cmidrule(lr){2-4}
& $\Omegam$ & log-uniform & [1/5, 1/2) & see \autoref{fig:sobol} \\
\cmidrule(lr){2-4}
& $\Omegab / \Omegam$ & log-uniform & [1/8, 1/4) \\
\cmidrule(lr){2-4}
& $\Omega_k / (1 - \Omega_k)$ & uniform & [-1/3, 1/3) \\
\cmidrule(lr){2-4}
& $h$ & log-uniform & [0.5, 1) \\
\cmidrule(lr){1-4}
\GADGET\ & softening ratio & log-uniform & [1/50, 1/20) \\
\cmidrule(lr){1-5}
& mesh shape & log-uniform & [128, 512] \\
\cmidrule(lr){2-4}
\pmwd\ & mesh cell size in Mpc & log-uniform & [0.2, 5] & MC \\
\cmidrule(lr){2-4}
& number of time steps* & log-uniform & [10, 1000] \\
\cmidrule(lr){1-4}
training & loss grid offset in its own spacing & uniform & $[0, 1)^3$ \\
\bottomrule
\end{tabular}
\end{table*}


\vspace{1em}
\subsection{Parameters and Configurations}

We run 512 \GADGET\ simulations that cover a wide range of
configurations and cosmological parameters, as listed in
\autoref{tab:param}.
Each simulation has $128^3$ particles and writes 121 snapshots linearly
spaced in scale factor from $1/16 + \Delta a$ to $1 + \Delta a$, with
spacing $1/128$.
$\Delta a$ is an offset, common for all snapshots in each simulation,
that varies as in \autoref{tab:param} and helps to sample different
alignment between snapshot and start/stop times.

The curvature $\Omega_k$ is related to the separate universe simulation
\citep{LiEtAl2014, WagnerEtAl2015}.

We have verified IC generation on GPUs is deterministic though different
from those on CPUs, so we generate ICs for both training data and
training on GPUs.


\begin{figure*}
  \centering
  \includegraphics[width=0.9\linewidth]{sobol.pdf}
  \caption{Randomized Quasi-Monte Carlo (RQMC) configuration with
    scrambled Sobol sequence of 512 points in 9D.
    Lower triangular panels show the 2D projections and the diagonal
    panels are the 1D cumulative histograms.
    From left to right (top to bottom), we use each dimension of the
    sample to scale the parameters as ordered in \autoref{tab:param}.
    We use the \texttt{scipy.stats.qmc} package \citep{SciPy} to
    generate the Sobol sequence \citep{Sobol1967}, which uses the
    direction number from \citet{JoeKuo2008} and the Owen scrambling
    \citep{Owen1998}.
    We search among 65536 scrambling seeds to minimize the mixture
    discrepancy (a uniformity measure) proposed in \citet{Zhou2013MD}.
  }
  \label{fig:sobol}
\end{figure*}


\vspace{1em}
\subsection{Target Data for Training and Testing}

\citet{GADGET-4}

We configure \GADGET\ with high force and time integration accuracy
settings.
For gravitational force, we choose pure FMM with $p=5$ and opening angle
0.4.
%FIXME list other from Config.sh and param.txt


\vspace{1em}
\subsection{Training of SO neural networks}

Given 512 \GADGET\ simulations and 121 output snapshots each, in total
we have 61952 snapshots for training.
Following \autoref{tab:param}, in each iteration, we randomly sample the
snapshots, the number of time steps, and the mesh shape (or its cell
size), and configure \pmwd\ with the same initial coniditions,
cosmological parameters and other common configurations.
To balance the work of GPUs during training, in each iteration we use
the same snapshot $a$, number of time steps, and mesh shape on all GPU.

We perform parallel training using distributed GPU devices.
The training on each snapshot is performed on a single GPU device.
During each training step, the gradients are collected and averaged across
devices to update the SO parameters.

Hyperparameters:

\YZ{
We implement the neural networks with \texttt{Flax} \citep{flax2020github}, and
use the optimizers in \texttt{Optax}, part of the DeepMind JAX Ecosystem
\citep{deepmind2020jax}.
}


\vspace{1em}
\subsection{Symbolic Regression}

We generate a sample of (input, output) data with the trained neural network to
perform feature selection and SR.

We use the random forest regressor for feature selection using impurity-based
importance.
\YZ{Maybe we can add another column in \autoref{tab:feat} for feature
importance?}
We use the \texttt{scikit-learn} \citep{scikit-learn} code for this.
% more details on the hyperparams in random forest

The selected important features are then used for SR.

We use the \texttt{PySR} \citep{Cranmer2023} SR code to fit for analytic
function expressions of the trained SO neural networks.

\YL{It is interesting to symbolically regress $f$, $k_i f$, and $f g$
etc to see which one gives simpler expression, given the possible
different rankings by complexity.
For example, $k_i f(k_i)$ may have a simpler form as a nonlinear odd
function.}


\vspace{1em}
\textit{\large Acknowledgements:}
YZ and YL were supported by The Major Key Project of PCL.
The Flatiron Institute is supported by the Simons Foundation.


\vspace{1em}
\textit{\large Code Availability:}
\pmwd\ \citep{Li2022b} is open-source on GitHub
\href{https://github.com/eelregit/pmwd}{\faGithub}, including the source
files and scripts of this paper
\href{https://github.com/eelregit/pmwd/tree/master/docs/papers/sto}{\faFile}.



\bibliographystyle{aasjournal}
\bibliography{sto}



%\listofchanges



\end{document}
