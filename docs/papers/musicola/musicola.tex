\documentclass[modern, trackchanges, dvipsnames]{aastex631}
  \urlstyle{sf}


\usepackage{CJK}
\usepackage{microtype}
\usepackage{savesym}
  \savesymbol{lambdabar}
\usepackage{newpxtext, eulerpx}
  \restoresymbol{newpx}{lambdabar}
\usepackage[T1]{fontenc}
\usepackage{fontawesome}
\usepackage{amsmath}
%  \allowdisplaybreaks
\usepackage{bm}
\usepackage{nicefrac}
\usepackage[caption=false]{subfig}
\usepackage[figure,figure*]{hypcap}
\usepackage{tikz}
  \usetikzlibrary{positioning, fit, calc, arrows.meta}
\usepackage{booktabs}


\newcommand{\pmwd}{{\usefont{T1}{nova}{m}{sl}pmwd}}

\newcommand{\gkai}[1]{\begin{CJK*}{UTF8}{gkai}\raisebox{.1em}{(}#1\raisebox{.1em}{)}\end{CJK*}}

\renewcommand{\sectionautorefname}{Sec.}
\renewcommand{\subsectionautorefname}{Sec.}
\renewcommand{\appendixautorefname}{App.}
\renewcommand{\figureautorefname}{Fig.}
\newcommand{\subfigureautorefname}{\figureautorefname}

\newcommand{\deltaD}{\delta^\textsc{d}}
\newcommand{\deltaK}{\delta^\textsc{k}}
\renewcommand{\d}{d}
\newcommand{\p}{\partial}
\newcommand{\cJ}{\mathcal{J}}
\newcommand{\cR}{\mathcal{R}}
\newcommand{\cL}{\mathcal{L}}
\newcommand{\cH}{\mathcal{H}}
\bmdefine{\vzero}{0}
\bmdefine{\vI}{I}
\bmdefine{\vnabla}{\nabla}
\bmdefine{\vtheta}{\theta}  % parameters
\bmdefine{\vomega}{\omega}  % white noise modes
\bmdefine{\vk}{k}  % wavevectors
\bmdefine{\vx}{x}  % comoving and canonical coordinates
\bmdefine{\vq}{q}  % Lagrangian coordinates
\bmdefine{\vs}{s}  % displacements
\bmdefine{\vp}{p}  % canonical momenta
\bmdefine{\va}{a}  % accelerations
\bmdefine{\vz}{z}  % states
\bmdefine{\vf}{f}
\bmdefine{\vF}{F}
\bmdefine{\vDelta}{\Delta}
\bmdefine{\vLambda}{\Lambda}
\bmdefine{\vlambda}{\lambda}
\bmdefine{\vvarphi}{\varphi}
\bmdefine{\vxi}{\xi}  % x adjoint
\bmdefine{\vpi}{\pi}  % p adjoint
\bmdefine{\valpha}{\alpha}  % force vjp x gradient
\bmdefine{\vzeta}{\zeta}  % force vjp theta gradient
\newcommand{\half}{\nicefrac12}
\newcommand{\As}{A_\mathrm{s}}
\newcommand{\ns}{n_\mathrm{s}}
\newcommand{\Omegam}{\Omega_\mathrm{m}}
\newcommand{\Omegab}{\Omega_\mathrm{b}}
\newcommand{\Mpc}{\mathrm{Mpc}}
\newcommand{\ic}{\mathrm{i}}
\newcommand{\Plin}{P_\mathrm{lin}}

\newcommand{\GPU}{NVIDIA A100 80GB SXM4}

\newcommand{\HL}[1]{\textcolor{Bittersweet}{#1}}



\begin{document}



\title{\large Perfectly Parallel Cosmological Simulation Embedded in Nested Perturbative Fields
\vspace{0.3em}}


\author[0000-0000-0000-0000]{\normalsize X}
\affiliation{zzz}
%
%\author[0000-0002-0701-1410]{\normalsize Yin Li \gkai{李寅}}
%\affiliation{Department of Mathematics and Theory, Peng Cheng
%Laboratory, Shenzhen, Guangdong 518066, China}
%
%\author[0000-0001-5044-7204]{\normalsize Drew Jamieson}
%\affiliation{Max Planck Institute for Astrophysics, 85748 Garching bei
%M\"unchen, Germany}
%
%\author[0000-0002-9300-2632]{\normalsize Yucheng Zhang \gkai{张宇澄}}
%\affiliation{Department of Mathematics and Theory, Peng Cheng
%Laboratory, Shenzhen, Guangdong 518066, China}


\shorttitle{musicola: N-body in Nested LPTs}
\shortauthors{X, Y, \& Z et al.}


\correspondingauthor{X \& Y}
\email{eelregit@gmail.com \& X}



\begin{abstract}

musicola = MUSIC + sCOLA, an approximate perfectly parallel method.

\end{abstract}



\section*{TODO}
\begin{itemize}
\item PDE solver: spectral vs finite difference?
\item boundary conditions: unbounded, Dirichlet, or?
\item Host memory stores the hi-res ICs, every GPU stores the lowest
  resolution ICs, and receives nested ICs of other resolutions before
  every run.
\end{itemize}



\vspace{1em}
\section{Introduction}


\vspace{1em}
\section{Methods}
\label{sec:method}


\vspace{1em}
\section{Results}


\vspace{1em}
\section{Conclusions}



\vspace{1em}
\textit{\large Acknowledgements:}
%YL and YZ were supported by The Major Key Project of PCL.


\vspace{1em}
\textit{\large Code Availability:}
\pmwd\ is open-source on GitHub
\href{https://github.com/eelregit/pmwd}{\faGithub}, including the source
files and scripts of this paper
\href{https://github.com/eelregit/pmwd/tree/master/docs/papers/musicola}{\faFile}.


\vspace{1em}
\textit{\large Software:}
\texttt{JAX} \citep{JAX}, \texttt{NumPy} \citep{NumPy}, \texttt{SciPy}
\citep{SciPy}, and \texttt{matplotlib} \citep{matplotlib}.



\vspace{1em}
\appendix


\vspace{1em}
\section{XYZ}
\label{app:xyz}



\bibliographystyle{aasjournal}
\bibliography{musicola}



%\listofchanges



\end{document}
