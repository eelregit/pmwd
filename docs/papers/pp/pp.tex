\documentclass[a4paper]{article}


\usepackage{CJK}
\usepackage{microtype}
%\usepackage{savesym}
%  \savesymbol{lambdabar}
\usepackage{newpxtext,eulerpx}
%  \restoresymbol{newpx}{lambdabar}
\usepackage[T1]{fontenc}
\usepackage[margin=3cm]{geometry}
%\usepackage{setspace}
\usepackage{amsmath}
%  \allowdisplaybreaks
\usepackage{bm}
\usepackage{dsfont}
%\usepackage{mathrsfs}
%\usepackage{soul}
%\usepackage{tensor}
\usepackage{graphicx}
%\usepackage{booktabs}
%\usepackage{placeins}
%\usepackage{subcaption}
%\usepackage[caption=false]{subfig}  % revtex hates caption
\usepackage{hyperref}
  \hypersetup{
    colorlinks=true,
    linkcolor=BrickRed,
    filecolor=BrickRed,
    citecolor=LimeGreen,
    urlcolor=Cerulean,
  }
\usepackage{natbib}
\usepackage[dvipsnames,svgnames,x11names]{xcolor}
\usepackage{comment}
  \newif\ifshow
  %\showtrue  % show
  \showfalse  % hide
  %
  \ifshow
    \includecomment{hide}
  \else
    \excludecomment{hide}
  \fi
\usepackage{tikz}
  \usetikzlibrary{positioning, fit, calc, arrows.meta}


\newcommand{\pmwd}{{\usefont{T1}{nova}{m}{sl}pmwd}}

\newcommand{\gkai}[1]{\begin{CJK*}{UTF8}{gkai}\raisebox{.1em}{(}#1\raisebox{.1em}{)}\end{CJK*}}

\renewcommand{\sectionautorefname}{Sec.}
\renewcommand{\subsectionautorefname}{Sec.}
\renewcommand{\appendixautorefname}{App.}
\renewcommand{\figureautorefname}{Fig.}
\newcommand{\subfigureautorefname}{\figureautorefname}

\DeclareMathOperator*{\E}{\mathds{E}}
\DeclareMathOperator{\tsum}{{\textstyle\sum}}
\newcommand{\1}{\mathds{1}}
\newcommand{\order}{\mathcal{O}}
\newcommand{\deltaD}{\delta^\textsc{d}}
\newcommand{\deltaK}{\delta^\textsc{k}}
\renewcommand{\d}{d}
\newcommand{\p}{\partial}
\newcommand{\cJ}{\mathcal{J}}
\newcommand{\cR}{\mathcal{R}}
\newcommand{\cL}{\mathcal{L}}
\newcommand{\cH}{\mathcal{H}}
\bmdefine{\vzero}{0}
\bmdefine{\vI}{I}
\bmdefine{\vnabla}{\nabla}
\bmdefine{\vtheta}{\theta}  % parameters
\bmdefine{\vomega}{\omega}  % white noise modes
\bmdefine{\vk}{k}  % wavevectors
\bmdefine{\vn}{n}  % wavevector indices
\bmdefine{\vx}{x}  % comoving and canonical coordinates
\bmdefine{\vq}{q}  % Lagrangian coordinates
\bmdefine{\vs}{s}  % displacements
\bmdefine{\vp}{p}  % canonical momenta
\bmdefine{\va}{a}  % accelerations
\bmdefine{\vz}{z}  % states
\bmdefine{\vf}{f}
\bmdefine{\vF}{F}
\bmdefine{\vvarphi}{\varphi}
\newcommand{\As}{A_\mathrm{s}}
\newcommand{\ns}{n_\mathrm{s}}
\newcommand{\Omegam}{\Omega_\mathrm{m}}
\newcommand{\Omegab}{\Omega_\mathrm{b}}
\newcommand{\Mpc}{\mathrm{Mpc}}
\newcommand{\ic}{\mathrm{i}}
\newcommand{\knyq}{k_\mathrm{Nyq}}
\newcommand{\kfun}{k_\mathrm{fun}}

\newcommand{\GPU}{NVIDIA A100 80GB SXM4}

\newcommand{\YL}[1]{\textcolor{Bittersweet}{#1}}



\begin{document}



\title{Patching Particle-Particle Particle-Mesh Method}


\author{Yin Li \gkai{李寅}}


\date{2022-12}


\maketitle



\begin{abstract}
Adding the PP compensation to the PM force.
\end{abstract}



\section{Introduction}


\section{Methods}


\subsection{Particle-Mesh Force}


\YL{Alex suggested starting from 1D.}

\YL{Describe briefly normal PM here. Also introduce possible filter
below. Explain that we are not doing splitting the usual way.}


\subsubsection{Partial Sum}

In a periodic box of size $L_1 \times \cdots \times L_d$ in
$\mathbb{R}^d$, Poisson's equation,
%
\begin{equation}
\nabla^2 u = f,
\end{equation}
%
becomes
%
\begin{equation}
- k_\vn^2 u(\vk_\vn) = f(\vk_\vn),
\end{equation}
%
in Fourier space, with discretized wavevectors $\vk_\vn \triangleq (2\pi
n_1 / L_1, \cdots, 2\pi n_d / L_d )^\intercal$ and integrer $n_i \in
\mathbb{Z}/N_i\mathbb{Z}$.
And the force $- \vnabla u$ becomes $i \vk_\vn f / k_\vn^2$.

At finite resolution of the mesh cell size $l$, the bandwidth is
truncated at the Nyquist frequency $\knyq = \pi / l$.
In a 1D periodic length $L$ with a mesh of size $N = L / l$, the
partial sum is equivalent to a convolution by (the Dirichlet kernel)
%
\begin{align}
\frac1{2 L} \Bigg(
  \sum_{n=-\lfloor \frac{N}2\rfloor}^{\lfloor \frac{N}2\rfloor}
  + \sum_{n=-\lfloor \frac{N-1}2\rfloor}^{\lfloor \frac{N-1}2\rfloor}
\Bigg)
\exp\bigl( i n\kfun x \bigr)
%
&= \frac{
  \sin \bigl[
    \bigl( \bigl\lfloor\frac{N}2\bigr\rfloor + \frac12 \bigr) \kfun x
  \bigr]
  + \sin \bigl[
    \bigl( \bigl\lfloor\frac{N-1}2\bigr\rfloor + \frac12 \bigr) \kfun x
  \bigr]
}{2 \sin \bigl( \frac12 \kfun x \bigr) L} \nonumber\\
%
&= \frac{\sin \bigl( \knyq x \bigr)}
        {\sin \bigl( \frac12 \kfun x \bigr) L}
  \cdot \begin{cases}
    \cos \bigl( \frac12 \kfun x \bigr) & \text{if } N \text{ is even}, \\
    1 & \text{if } N \text{ is odd}.
  \end{cases}
\end{align}
%
$\kfun = 2\pi / L$ is the fundamental frequency, and the first sum
includes the symmetrized Nyquist mode for even $N$.

To preserve Hermiticity, for even $N$ we truncate the force even before
$\kfun$, due to the sign ambiguity of the wavevector at the Nyquist
frequency.
%
\begin{equation}
\frac1L
  \sum_{n=-\lfloor \frac{N-1}2\rfloor}^{\lfloor \frac{N-1}2\rfloor}
  \exp\bigl( i n\kfun x \bigr)
%
= \frac{\sin \bigl[
    \bigl( \bigl\lfloor\frac{N-1}2\bigr\rfloor + \frac12 \bigr) \kfun x
  \bigr]
}{\sin \bigl( \frac12 \kfun x \bigr) L}
\end{equation}


\YL{TODO: verify that in 1D the partial sum explains the force error.
Two ways: mass integral vs force partial sum, off by the Nyquist?}


\YL{Minimize force error above a certain radius, e.g. beyond 27 cells,
(in a way similar to choosing NUFFT kernel?):
\begin{itemize}
\item \url{https://arxiv.org/abs/1712.04732},
\item \url{https://www.frontiersin.org/articles/10.3389/fphy.2016.00028/full},
\item \url{https://journals.aps.org/pre/pdf/10.1103/PhysRevE.95.063303}.
\end{itemize}
}


\YL{Filter/window design: Jupyter notebook with interactive sliders,
\href{https://docs.scipy.org/doc/scipy/reference/signal.windows.html}{scipy.signal.windows},
\href{https://docs.scipy.org/doc/scipy/reference/signal.html}{scipy.signal}.
}


\YL{2 possible ways to optimize:
suppress the oscillation outside of the PP patch size;
shift the phase of the oscillation so that there are both positive and
negative parts between two nearby mesh points.
}


\subsubsection{Equivalent Point Source}

simple cubic (CIC), bcc, interlocking/interlacing cubic.
\YL{See \citet{HockneyEastwood1988} for the interlacing one.
Other references:
\begin{itemize}
\item \url{https://aip.scitation.org/doi/10.1063/1.3430521},
\item \url{https://aip.scitation.org/doi/pdf/10.1063/1.3430521},
\item \url{https://aip.scitation.org/doi/pdf/10.1063/1.3657407}.
\end{itemize}
}

%* Suppose the two interlacing meshes are A and B, is the following a way
%  to have 16 equivalent grid points (up to octopole)?
%  1. src: (A + shifted B) at A, tgt: A
%  2. src: (B + shifted A) at B, tgt: B


\subsection{Particle-Particle Force}



\subsection{Force Patching}


\YL{Quantify force error including (position-dependent) magnitude and
volume.}


For cosmological applications, we want to optimize for point clouds that
are uniform (and Gaussian with known power spectrum) on large scales.
\YL{Complexity analysis given variance of particle density.}


\YL{Optimal mesh cell size given variance of particle density for fixed
error budget, cosmology and time dependence.}



\subsection{Complexity}

$N$ particles, $M$ cells, volume $V$, mean matter density
$\bar\rho$, particle mass $m = \bar\rho V / N$, density field $\rho(\vx)
= \bar\rho [1 + \delta(\vx)]$, $C_i$ is the $i$-th cell, and $D_i$ are
the cells whose particles directly interact with those in $C_i$ (so $C_i
\subsetneq D_i$).
Each cell $C$ has volume $V / M$, and each direct interacting region $D$
has volume $d V / M$ with $d$ being an integer greater than 1.
The total number of pairs of interactions is
%
\begin{align}
\tsum_{i=1}^M \E \bigl[ \tsum_{p\in C_i} \1_{p'\in D_i} \bigr]
%
&\simeq \frac{N^2}{V^2} \tsum_{i=1}^M \E \biggl[
  \int_{C_i} \!\!\!\d\vx \int_{D_i} \!\!\!\d\vx'\,
  [1 + \delta(\vx)] [1 + \delta(\vx')] \biggr] \nonumber\\
%
&= \frac{M N^2}{V^2} \int_C \!\!\d\vx \int_D \!\!\d\vx'\,
  \bigl[ 1 + \xi(\vx' - \vx) \bigr]
= \frac{d N^2}M  \bigl( 1 + \sigma^2_{CD} \bigr),
\end{align}
%
where $\xi$ is the matter density 2-point correlation function, and we
have replaced the sums over particles with the integrals over the
density fields, $\sum_p \to m^{-1} \int \!\d\vx\, \rho(\vx) = N V^{-1}
\int \!\d\vx\, [1 + \delta(\vx)]$, in deriving the first equality.
We have also used the density covariance between $C$ and $D$, defined as
%
\begin{equation}
\sigma^2_{CD} \triangleq \int \!\!\frac{\d\vk}{(2\pi)^3}\,
  P(k) W_C^*(\vk) W_D(\vk)
= \frac{M^2}{d V^2} \int_C \!\!\d\vx \int_D \!\!\d\vx'\,
  \xi(\vx' - \vx),
\end{equation}
%
where $P(k)$ is the matter density power spectrum and the Fourier
transform of $\xi$.
$\sigma^2_{CD}$ depends on the geometry of $C$ and $D$, through their
Fourier space windows:
%
\begin{align}
W_C(\vk) &\triangleq \frac{M}V \int_C \!\!\d\vx\, e^{i\vk\cdot\vx},
  \nonumber\\
W_D(\vk) &\triangleq \frac{M}{dV} \int_D \!\!\d\vx\, e^{i\vk\cdot\vx},
\end{align}
%
both normalized to 1 at $\vk=\vzero$.



\section{Results}


\section{Conclusions}






\bibliographystyle{plainnat}
\bibliography{pp}



\end{document}
